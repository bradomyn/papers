\section{Introduction}

Synchronization is one of the most important feature in control and timing
systems. The performance depends on the synchronization accuracy between 
the networking elements throughout the network, especially in systems 
where time-critical events are accurately scheduled to be executed in
distributed nodes. Such systems are quite common in particle accelerators
facilities, like the new timing system at GSI~\cite{biblio:FAIRtimingSystem} and 
CERN~\cite{biblio:cernWr}. The chosen technology  in both facilities is based on
the White Rabbit project. White Rabbit (WR) provides the technology and techniques to create a 
reliable and robust data and timing network with low-latency, deterministic packet 
delivery and synchronization. WR is based on the standards Synchronous Ethernet 
(SyncE)~\cite{biblio:synch} and  Ethernet (IEEE~802.3)~\cite{biblio:ethernet}. Besides, it extends IEEE~1588 
(PTP)~\cite{biblio:ptp} for achieving sub-nanoseconds  accuracy and picoseconds precision.
Besides GSI and CERN, in the WR project~\cite{biblio:wrproj} there are other 
institutes and experiments (LHAASO, KM3NeT etc...) that are taking an active part in 
development and adoption of the technology, as well as commercial companies (Seven Solutions, 
Integrasys, Elproma, Creotech, National Instruments etc..).

So far in WR, the synchronization is propagated in a hierarchical topology, from the master clock, 
WR Master (master), to the slave clocks, WR Slave (slave), using White Rabbit Switches (WRS). 
A WRS, in terms of PTP, is a two-step BC using the delay request-respond
mechanism (DDR) ~\cite{biblio:ptp}  
for the synchronization. In this paper, the author proposes an implementation
of a Transparent Clock (TC) using the existing WR extension to PTP, WRPTP~\cite{biblio:wrptp} 
and WR hardware~\cite{biblio:spec},~\cite{biblio:wrswitch}. 

Among the advantages of the Transparent Clocks (TC) for large timing networks
(e.g more than 2000 slaves in GSI), it is especially interesting
for the WR project the Peer-to-Peer TCs. They are suitable for timing
systems that require high resilience in the event of changes in the topology 
(e.g. switch failure), since the delay measurement is already available for
the synchronization in the new link, in contrast to Boundary Clocks (BC) that
have to calculate the delay once the new link is active. Another attractive feature 
of TCs for a WR timing system is the measurement of the residence time in the network 
devices which eliminates the error that results from queuing delays. 

In this paper, the author describes the WRPTP and WR synchronization (Section
II), and how the extension can be adopted by E2E and P2P TC (Section III). Then,
the author presents issues of implementing a WR TC  and an estimation of the performance.




