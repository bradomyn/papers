\section{Introduction}

Synchronization is one of the most important feature in control and timing
systems. The performance depends on the synchronization accuracy between 
the networking elements throughout the network, specially in systems 
where time-critical events are accurately scheduled to be executed in
distributed nodes. Such systems are quite common in facilities like particle 
accelerators like GSI and CERN. WR is the chosen technology for the timing system 
at GSI ~\cite{biblio:FAIRtimingSystem} and CERN ~\cite{biblio:cernWr}. 
Besides these two facilities, in the WR project ~\cite{biblio:wrproj}, there are others 
instituted and experiments (LHAASO, KM3NeT etc...) that are taking an active part in 
development and adoption of the technology, as well as commercial companies (Seven Solutions, 
Integrasys, Elproma, Creotech, National Instruments etc..).

White Rabbit (WR) provides the technology and techniques to create a 
reliable and robust data and timing network with low-latency, deterministic packet 
delivery and synchronization. WR is based on the standards Synchronous Ethernet 
(SyncE) ~\cite{biblio:synch} and  Ethernet (IEEE~802.3) ~\cite{biblio:ethernet}. Besides, it extends IEEE~1588 
(PTP)~\cite{biblio:ptp} for achieving sub nanoseconds  accuracy and picoseconds precision.

So far in WR, the synchronization is propagated in a hierarchical topology, from the master clock, 
WR Master (master), to the slave clocks, WR Slave (slave), using White Rabbit Switches (WRS). 
A WRS, in terms of PTP, is a two-step boundary clocks using the delay request-respond mechanism ~\cite{biblio:ptp}  
for the synchronization. In this paper, the author proposes an implementation
of a Transparent Clock (TC) using the existing WR extension to PTP, WRPTP ~\cite{biblio:wrptp} 
and WR hardware ~\cite{biblio:spec},~\cite{biblio:wrswitch}, and evaluates the
WR TC taking into account the principle features that characterizes the performance of a TC. 


