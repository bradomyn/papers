\section{Introduction}

Synchronization is an important feature in control and timing
systems. The performance depends on the synchronization accuracy between 
the networking elements throughout the network, specially in systems 
where time-critical events are accurately scheduled to be executed in
distributed nodes. 

White Rabbit (WR) technology provides allows to create a reliable and robust Ethernet network with low-latency, 
deterministic packet delivery and synchronization. WR is based on the standards:Synchronous Ethernet 
(SyncE) ~\cite{biblio:synch}, Ethernet (IEEE~802.3) ~\cite{biblio:ethernet}. Besides, it extends IEEE~1588 
(PTP)~\cite{biblio:ptp} for achieving subnanoseconds accuracy and picoseconds precision.
WR will be the control and timing system at GSI ~\cite{biblio:FAIRtimingSystem} and CERN ~\cite{biblio:cernWr}. 
There are others instituted and facilities  (LHAASO, KM3NeT etc...) that are taking an active part in 
the WR Project ~\cite{biblio:wrproj}, as well as commercial companies (Seven Solutions, Integrasys, Elproma, 
Creotech, National Instruments etc..).

So far in WR, the synchronization is propagated from the Master Clock, WR Master, to the
Slaves Clock, WR Slave, in the network using White Rabbit Switches (WRS). A WRS,
in terms of PTP, is a two-step Boundary Clocks using the Delay Request-Respond Mechanism ~\cite{biblio:ptp}  
for the synchronization. In this paper, the author proposes an implementation
of a Transparent Clock (TC) using the existing WR extension to PTP, WRPTP ~\cite{biblio:wrptp} 
and the existing hardware with WR support ~\cite{biblio:spec},~\cite{biblio:wrswitch}, 
and evaluates the principle features that characterizes the performance of a TC. 

Since GSI will adopt WR technology for the new control and timing system, the
author presents a timing topology making use of TC based on WR. 

